\documentclass[a4paper, 11pt]{article}

\usepackage[margin=3cm]{geometry}
\geometry{a4paper, top=2cm, bottom=3cm, left=1.5cm, right=1.5cm, heightrounded, bindingoffset=5mm}

\usepackage[T1]{fontenc}
\usepackage[utf8]{inputenc}
\usepackage[italian]{babel}

\usepackage{psfrag,amsmath,amssymb,amsthm,amsfonts,verbatim}
\usepackage{mathtools, physics, textcomp}
\usepackage[figurename=Fig.,font=small,labelfont=bf]{caption}
\usepackage{capt-of, subcaption, wrapfig, float, hyperref, cleveref}

\usepackage{latexsym, gensymb, graphicx, graphics}
\usepackage{xcolor,color,soul}
\usepackage{fancyhdr}
\usepackage{indentfirst}
\usepackage{newlfont}
\usepackage{footnote}

\setlength{\parindent}{0pt}

%\usepackage{tikz}
%\usetikzlibrary{snakes}
%\usetikzlibrary{positioning}
%\usetikzlibrary{shapes,arrows}

%\tikzstyle{block} = [draw, fill=white, rectangle, 
%	minimum height=3em, minimum width=6em]
%\tikzstyle{sum} = [draw, fill=white, circle, node distance=1cm]
%\tikzstyle{input} = [coordinate]
%\tikzstyle{output} = [coordinate]
%\tikzstyle{pinstyle} = [pin edge={to-,thin,black}]

\graphicspath{ {../attachments/} }

\newcommand{\bb}{
	\bigbreak
}

\newcommand{\R}{
	\ensuremath
	\mathbb{R}
}

\newcommand{\C}{
	\ensuremath
	\mathbb{C}
}

\newcommand{\Lap}{
	\ensuremath
	\mathcal{L}
}

\newcommand{\M}[3]{
	\ensuremath
	\mathcal{M}\paren{#1,#2,#3}
}

\newcommand{\llimit}[3]{
	\ensuremath
	\lim_{#1 \rightarrow #2} #3
}

\newcommand{\pinf}{
	\ensuremath
	+\infty
}

\newcommand{\minf}{
	\ensuremath
	-\infty
}

\newcommand{\intreal}[2]{
	\ensuremath
	\int_{\R} #1 \dd{#2}
}

\newcommand{\intnoreal}[4]{
	\ensuremath
	\int_{#1}^{#2} #3 \dd{#4}
}

\newcommand{\ifincases}[1]{
	\ensuremath
	\; \textrm{ se } \; #1
}

\newcommand{\sgn}[1]{
	\operatorname{sgn} \left( #1 \right)
}

\newcommand{\paren}[1]{
	\ensuremath
	\left( #1 \right)
}

\newcommand{\sparen}[1]{
	\ensuremath
	\left[ #1 \right]
}

\newcommand{\resource}[3]{
	\begin{center}
	\includegraphics[scale = #1]{#2}
	\end{center}
	\captionof{figure}{#3}
}

\newcommand{\rarr}{
	$\longrightarrow$
}

\newcommand{\lt}[1]{
	\ensuremath
	\mathcal{L}\sparen{#1}
}

\newcommand{\lat}[1]{
	\ensuremath
	\mathcal{L}^{-1}\sparen{#1}
}

\newcommand{\ltint}[3]{
	\ensuremath
	\int_{0^-}^{\pinf} #1 e^{#2} \dd{#3}
}

\newcommand{\latint}[3]{
	\ensuremath
	\frac{1}{2\pi j}\int_{\sigma-j\infty}^{\sigma + j\infty} #1 e^{#2} \dd{#3}
}

\newcommand{\cof}[1]{
	\ensuremath
	\textrm{Cof}\paren{#1}
}
\usepackage{relsize}

\newcommand{\courseacronym}{CAT}
\newcommand{\coursename}{Controlli Automatici - T}
\newcommand{\tipology}{\textbf{a} }
\renewcommand{\trace}{\textbf{3}}
\newcommand{\projectname}{\textbf{\textit{Controllo dell'assetto di un drone planare}}}
\newcommand{\group}{\textbf{AA}}

\title{ \vspace{-1in}
		\huge \strut \coursename \strut \\
		\Large  \strut Progetto Tipologia \tipology - Traccia \trace \\
		\Large  \strut \projectname\strut \\
		\Large  \strut Gruppo \group\strut
		\vspace{-0.4cm}
}
\author{Autori: \textbf{Leonardo Dominici, Fabio Colonna, Edoardo Marinelli, Emanuele Di Luzio}}
\date{Gennaio 2023}

\begin{document}

\maketitle
%\vspace{-0.5cm} this is unnecessary

Il progetto riguarda il controllo dell'\textit{assetto di un drone planare}, la cui dinamica viene descritta dalle seguenti equazioni differenziali 
%
\begin{subequations}\label{eq:system}
\begin{align}
	\dot \theta & = \omega 
	\\
	J \dot \omega & = -\beta \omega + \frac{a}{2} \sin(\theta) F_v + a F_p
\end{align}
\end{subequations}
%
dove $J \in \R$ rappresenta il \textbf{momento di inerzia} del drone rispetto all'asse di rotazione che passa per il baricentro, $\beta \in \R$ il \textit{coefficiente di attrito dinamico} dovuto alla presenza dell'aria, $a \in \R$ la \textbf{semi-ampiezza planare} del drone, e $F_v \in \R$ la \textbf{forza costante dovuta all'azione del vento}.

\bb

La variabile di ingresso del sistema, definita come $F_p(t) = F_1(t) - F_2(t)$, indica la \textbf{differenza tra le forze di propulsione} $F_1(t), F_2(t)$ applicate sul drone. Si suppone, infine, che sia possibile \textbf{misurare l'angolo di inclinazione} $\theta(t)$ per ogni istante di tempo.

\resource{0.5}{drone.png}{Rappresentazione nel piano del drone considerato.}

\section{Passaggio alla forma di stato e linearizzazione}
\subsection{Forma di stato}

Per farlo, individuiamo la composizione dello stato $x(t)$,  partendo dalla relazione differenziale \eqref{eq:system} modificata in modo da isolare i termini che compaiono derivati:
\begin{align*}
	\begin{aligned}
		\dot \theta & = \omega \\
		\dot \omega & = -\frac{\beta}{J} \omega + \frac{a}{2J} \sin(\theta) F_v + \frac{a}{J} F_p
	\end{aligned}
\end{align*}
Prendendo in considerazione anche le informazioni date dalle specifiche riguardo l'ingresso ($F_p$) e l'uscita ($\theta$, in quanto supposta misurabile), definiamo:
\begin{align*}
	x \coloneqq \begin{bmatrix}
		x_1 \\ x_2
	\end{bmatrix} = \begin{bmatrix}
		\theta \\ \omega
	\end{bmatrix} \quad \quad u \coloneqq F_p \quad \quad y \coloneqq x_1
\end{align*}
Mettendo insieme quanto ricavato, scriviamo la relazione in forma di stato (i.e. $\dot x = f(x, u)$ e $y = h(x, u)$):
\begin{subequations}\label{eq:fds}
\begin{align}
	\dot x & = \begin{bmatrix}
		\dot x_1 \\ \dot x_2
	\end{bmatrix} = \begin{bmatrix} 
		f_1(x, u) \\ f_2(x, u)
	\end{bmatrix} = \begin{bmatrix}
		x_2 \\ -\frac{\beta}{J} x_2 + \frac{a}{2J} \sin(x_1)F_v + \frac{a}{J} u
	\end{bmatrix} \\
	y & = h(x, u) =  x_1
\end{align}
\end{subequations}
Avendo uno stato in $\R^2$ e ingresso e uscita entrambi \textbf{scalari}, il sistema in analisi è \textbf{SISO}.

\subsection{Coppia di equilibrio e linearizzazione intorno ad essa}
Chiamiamo $(x_e, u_e)$ una \textbf{coppia di equilibrio} se applicando un ingresso $u_e$ in presenza di uno stato $x_e$, il sistema rimane \textit{piantato} nello stato in cui si trova. Da qui concludiamo che \textit{variazione dello stato} è nulla, e che quindi è possibile trovare la coppia di equilibrio ponendo $f(x, u) = \dot x = 0$:
\begin{align*}
	\begin{bmatrix}
		f_1(x_e, u_e) \\ f_2 (x_e, u_e)
	\end{bmatrix} = \begin{bmatrix}
		x_{2,e} \\ -\frac{\beta}{J} x_{2,e} + \frac{a}{2J} \sin(x_{1,e})F_v + \frac{a}{J} u_e
	\end{bmatrix} = \begin{bmatrix}
		0 \\ 0
	\end{bmatrix}
\end{align*}
Facendo riferimento alle specifiche, il parametro $\theta_e$ è conosciuto ed è pari a $\pi/6$, e $F_v = -2$. In forma di stato, dunque, otteniamo una nuova relazione che possiamo utilizzare per trovare la coppia ($\theta_e = x_{1_e}$):
\begin{align}\label[type]{eq:equilibrio}
	x_e = \begin{bmatrix}
		x_{1,e} \\ x_{2,e}
	\end{bmatrix} = \begin{bmatrix}
		\pi / 6 \\ 0
	\end{bmatrix} \quad \quad
   u_e & = - \frac{1}{2} \sin(x_{1,e})F_v = \sin(x_{1,e}) = 0.5
\end{align}
A questo punto siamo in grado di \textbf{linearizzare il sistema intorno alla coppia di equilibrio} appena calcolata.
%
L'evoluzione del sistema espressa nelle variabili alle variazioni pu\`o essere approssimativamente descritta mediante il seguente sistema lineare
%
\begin{align*}
	\delta \dot{x} &= A\delta x + B\delta u
	\\
	\delta y &= C\delta x + D\delta u
\end{align*}
%
dove le matrici $A$, $B$, $C$ e $D$ vengono calcolate come matrici jacobiane derivate dall'espressione tramite polinomio di Taylor delle equazioni di stato ed uscita per una traiettoria \textit{vicina} all'equilibrio. Dalle specifiche recuperiamo le seguenti relazioni: $\beta = a = J = 2$:
%
\begin{subequations}
\begin{align}\label{eq:matrices}
	A &= \pdv{f(x_e, u_e)}{x} = \eval{\begin{bmatrix}
		\pdv{f_1(x, u)}{x_1} & \pdv{f_1(x, u)}{x_2} \\
		\pdv{f_2(x, u)}{x_1} & \pdv{f_2(x, u)}{x_2}
	\end{bmatrix}}_{\substack{x = x_e \\ u = u_e}} = 
	\begin{bmatrix}
		0 & 1 \\
		\frac{a}{2J} \cos(x_{1,e})F_v & - \frac{\beta}{J}
	\end{bmatrix} =
	\begin{bmatrix}
		0 & 1 \\ -\frac{\sqrt{3}}{2} & -1
	\end{bmatrix}
\end{align}
\begin{align}
	B = \pdv{f(x_e, u_e)}{u} = \eval{\begin{bmatrix}
		\pdv{f_1(x, u)}{u}  \\
		\pdv{f_2(x, u)}{u}
	\end{bmatrix}}_{\substack{x = x_e \\ u = u_e}} = \begin{bmatrix}
		0 \\ \frac{a}{J}
	\end{bmatrix} = \begin{bmatrix}
		0 \\ 1
	\end{bmatrix}
\end{align}
%
\begin{align}
	C = \pdv{h(x_e, u_e)}{x} = \eval{\begin{bmatrix}
		\pdv{h(x, u)}{x_1} & \pdv{h(x, u)}{x_2}
	\end{bmatrix}}_{\substack{x = x_e \\ u = u_e}} = \begin{bmatrix}
		1 & 0
	\end{bmatrix}
\end{align}
\begin{align}
	D = \pdv{h(x_e, u_e)}{u} = \eval{\begin{bmatrix}
		\pdv{h(x, u)}{u}
	\end{bmatrix}}_{\substack{x = x_e \\ u = u_e}} = 0	
\end{align}
\end{subequations}
La dimensione delle matrici è coerente con la definizione di SISO. A questo punto, scriviamo le equazioni del sistema linearizzato in forma matriciale:
\begin{subequations}\label{eq:linearized_system}
	\begin{align}
		\delta \dot x & = \begin{bmatrix}
			0 & 1 \\ -\frac{\sqrt{3}}{2} & -1
		\end{bmatrix} \delta x + \begin{bmatrix}
			0 \\ 1
		\end{bmatrix} \delta u \\
		\delta y & = \begin{bmatrix}
			1 & 0
		\end{bmatrix} \delta x
	\end{align}
\end{subequations}

\section{Funzione di trasferimento}
Procediamo con il calcolo della FdT facendo riferimento alla definizione matriciale, ottenuta partendo dalla trasformata dell'evoluzione forzata dell'uscita del sistema lineare:
\begin{align*}
	G(s) = C(sI-A)^{-1}B +D = C\sparen{\frac{(sI-A)^\dag}{\det(sI-A)}}B + D
\end{align*}
dove $\dag$ è usato per indicare la \textbf{matrice aggiunta}, cioè la trasposta della matrice contenente i \textit{complementi algebrici} degli elementi originari. Definiamo la matrice in questione e calcoliamone subito il suo determinante:
\begin{align*}
	sI-A = \begin{bmatrix}
		s & -1 \\ \frac{\sqrt{3}}{2} & s+1
	\end{bmatrix} \quad \quad \quad \det(sI-A) = s^2+s +\frac{\sqrt{3}}{2}
\end{align*}
e procediamo con il calcolo dei coefficienti algebrici, ricordando che $\cof{a_{i,j}} = (-1)^{i+j}\det(M_{i,j})$, con $M$ matrice a cui è stata soppressa la i-esima riga e la j-esima colonna. Dunque, chiamando $k_{i,j}$ gli elementi della matrice $sI-A$, abbiamo:
\begin{align*}
	\cof{k_{1,1}} & = (-1)^{i+j} \det(s+1) = s+1 \quad & \quad \cof{k_{1,2}} & = -\det\paren{\frac{\sqrt{3}}{2}} = -\frac{\sqrt{3}}{2} \\
	\cof{k_{2,1}} & = -\det(-1) = 1 \quad & \quad \cof{k_{2,2}} & \det(s) = s 
\end{align*}
Otteniamo la matrice aggiunta eseguendo la trasposta di quella dei cofattori
\begin{align*}
	(sI-A)^\dag = \begin{bmatrix}
		s+1 & -\frac{\sqrt{3}}{2} \\ 
		1 & s
	\end{bmatrix}^T = \begin{bmatrix}
		s+ 1 & 1 \\
		-\frac{\sqrt{3}}{2} & s
	\end{bmatrix}
\end{align*}
Ricollegandoci alla definizione della $G$ ad inizio sezione, scriviamo:
\begin{align*}
	G(s) = \begin{bmatrix} 1 & 0 \end{bmatrix} \begin{bmatrix}
		\frac{s+1}{s^2+s + \frac{\sqrt{3}}{2}} & \frac{1}{s^2+s + \frac{\sqrt{3}}{2}} \\
		\frac{ -\frac{\sqrt{3}}{2}}{s^2+s + \frac{\sqrt{3}}{2}} & \frac{s}{s^2+s + \frac{\sqrt{3}}{2}}
	\end{bmatrix} \begin{bmatrix}
		0 \\ 1
	\end{bmatrix} = \frac{1}{s^2+s + \frac{\sqrt{3}}{2}}
\end{align*}
Scomponiamo il denominatore per evidenziare i poli, e otteniamo la relazione finale:
\begin{align}\label{eq:fdt}
	G(s) = \frac{1}{\sparen{s + \paren{\frac{1+\sqrt{1-2\sqrt{3}}}{2}}}\sparen{s + \paren{\frac{1-\sqrt{1-2\sqrt{3}}}{2}}}}
\end{align}

Dunque il sistema linearizzato \eqref{eq:linearized_system} è caratterizzato dalla funzione di trasferimento \eqref{eq:fdt} con guadagno $\mu=1$, 2 poli reali () e nessuno zero. In figura \dots mostriamo il corrispondente diagramma di Bode che però non farò stasera perché ho sonno regaz a domani :)


\section{Mappatura specifiche del regolatore}
\label{sec:specifications}

Le specifiche da soddisfare sono
\begin{itemize}
	\item[1)] \dots\\
	\item[2)] \dots\\
	....\\
	\item[6)] \dots.
\end{itemize}
%
Andiamo ad effettuare la mappatura punto per punto le specifiche richieste. \dots  

Pertanto, in Figura \dots, mostriamo il diagramma di Bode della funzione di trasferimento $G(s)$ con le zone proibite emerse dalla mappatura delle specifiche.

\dots

\section{Sintesi del regolatore statico}
\label{sec:static_regulator}

In questa sezione progettiamo il regolatore statico $R_s(s)$ partendo dalle analisi fatte in sezione~\ref{sec:specifications}.

\dots

Dunque, definiamo la funzione estesa $G_e(s) = R_s(s)G(s)$ e, in Figura \dots, mostriamo il suo diagramma di Bode per verificare se e quali zone proibite vengono attraversate.

\dots

Da Figura \dots, emerge \dots


\section{Sintesi del regolatore dinamico}

In questa sezione, progettiamo il regolatore dinamico $R_d(s)$. 
%
Dalle analisi fatte in Sezione~\ref{sec:static_regulator}, notiamo di essere nello Scenario di tipo \dots. Dunque, progettiamo $R_d(s)$ riccorrendo a \dots


In Figura \dots, mostriamo il diagramma di Bode della funzione d'anello $L(s) = R_d(s) G_e(s)$

\dots

\section{Test sul sistema linearizzato}

In questa sezione, testiamo l'efficacia del controllore progettato sul sistema linearizzato con \dots

\section{Test sul sistema non lineare}

In questa sezione, testiamo l'efficacia del controllore progettato sul modello non lineare con \dots


\section{Punti opzionali}

\subsection{Primo punto}

\dots 

\subsection{Secondo punto}

\dots

\subsection{Terzo punto}

\dots

\section{Conclusioni}

\dots

\end{document}
