\documentclass[a4paper, 11pt]{article}

\usepackage[margin=3cm]{geometry}
\geometry{a4paper, top=2cm, bottom=3cm, left=1.5cm, right=1.5cm, heightrounded, bindingoffset=5mm}

\usepackage[T1]{fontenc}
\usepackage[utf8]{inputenc}
\usepackage[italian]{babel}

\usepackage{psfrag,amsmath,amssymb,amsthm,amsfonts,verbatim}
\usepackage{mathtools, physics, textcomp}
\usepackage[figurename=Fig.,font=small,labelfont=bf]{caption}
\usepackage{capt-of, subcaption, wrapfig, float, hyperref, cleveref}

\usepackage{latexsym, gensymb, graphicx, graphics}
\usepackage{xcolor,color,soul}
\usepackage{fancyhdr}
\usepackage{indentfirst}
\usepackage{newlfont}
\usepackage{footnote}

\setlength{\parindent}{0pt}

%\usepackage{tikz}
%\usetikzlibrary{snakes}
%\usetikzlibrary{positioning}
%\usetikzlibrary{shapes,arrows}

%\tikzstyle{block} = [draw, fill=white, rectangle, 
%	minimum height=3em, minimum width=6em]
%\tikzstyle{sum} = [draw, fill=white, circle, node distance=1cm]
%\tikzstyle{input} = [coordinate]
%\tikzstyle{output} = [coordinate]
%\tikzstyle{pinstyle} = [pin edge={to-,thin,black}]

\graphicspath{ {../attachments/} }

\newcommand{\bb}{
	\bigbreak
}

\newcommand{\R}{
	\ensuremath
	\mathbb{R}
}

\newcommand{\C}{
	\ensuremath
	\mathbb{C}
}

\newcommand{\Lap}{
	\ensuremath
	\mathcal{L}
}

\newcommand{\M}[3]{
	\ensuremath
	\mathcal{M}\paren{#1,#2,#3}
}

\newcommand{\llimit}[3]{
	\ensuremath
	\lim_{#1 \rightarrow #2} #3
}

\newcommand{\pinf}{
	\ensuremath
	+\infty
}

\newcommand{\minf}{
	\ensuremath
	-\infty
}

\newcommand{\intreal}[2]{
	\ensuremath
	\int_{\R} #1 \dd{#2}
}

\newcommand{\intnoreal}[4]{
	\ensuremath
	\int_{#1}^{#2} #3 \dd{#4}
}

\newcommand{\ifincases}[1]{
	\ensuremath
	\; \textrm{ se } \; #1
}

\newcommand{\sgn}[1]{
	\operatorname{sgn} \left( #1 \right)
}

\newcommand{\paren}[1]{
	\ensuremath
	\left( #1 \right)
}

\newcommand{\sparen}[1]{
	\ensuremath
	\left[ #1 \right]
}

\newcommand{\resource}[3]{
	\begin{center}
	\includegraphics[scale = #1]{#2}
	\end{center}
	\captionof{figure}{#3}
}

\newcommand{\rarr}{
	$\longrightarrow$
}

\newcommand{\lt}[1]{
	\ensuremath
	\mathcal{L}\sparen{#1}
}

\newcommand{\lat}[1]{
	\ensuremath
	\mathcal{L}^{-1}\sparen{#1}
}

\newcommand{\ltint}[3]{
	\ensuremath
	\int_{0^-}^{\pinf} #1 e^{#2} \dd{#3}
}

\newcommand{\latint}[3]{
	\ensuremath
	\frac{1}{2\pi j}\int_{\sigma-j\infty}^{\sigma + j\infty} #1 e^{#2} \dd{#3}
}

\newcommand{\cof}[1]{
	\ensuremath
	\textrm{Cof}\paren{#1}
}

\newcommand{\courseacronym}{CAT}
\newcommand{\coursename}{Controlli Automatici - T}
\newcommand{\tipology}{\textbf{a} }
\renewcommand{\trace}{\textbf{3}}
\newcommand{\projectname}{\textbf{\textit{Controllo dell'assetto di un drone planare}}}
\newcommand{\group}{\textbf{AA}}

\title{ \vspace{-1in}
		\huge \strut \coursename \strut \\
		\Large  \strut Progetto Tipologia \tipology - Traccia \trace \\
		\Large  \strut \projectname\strut \\
		\Large  \strut Gruppo \group\strut
		\vspace{-0.4cm}
}
\author{Autori: \textbf{Leonardo Dominici, Fabio Colonna, Edoardo Marinelli, Emanuele Di Luzio}}
\date{Gennaio 2023}

\begin{document}

\maketitle
%\vspace{-0.5cm}

Il progetto riguarda il controllo dell'\textit{assetto di un drone planare}, la cui dinamica viene descritta dalle seguenti equazioni differenziali 
%
\begin{subequations}\label{eq:system}
\begin{align}
	\dot \theta & = \omega 
	\\
	J \dot \omega & = -\beta \omega + \frac{a}{2} \sin(\theta) F_v + a F_p
\end{align}
\end{subequations}
%
dove $J \in \R$ rappresenta il \textbf{momento di inerzia} del drone rispetto all'asse di rotazione che passa per il baricentro, $\beta \in \R$ il \textit{coefficiente di attrito dinamico} dovuto alla presenza dell'aria, $a \in \R$ la \textbf{semi-ampiezza planare} del drone, e $F_v \in \R$ la \textbf{forza costante dovuta all'azione del vento}.

\bb

La variabile di ingresso del sistema, definita come $F_p(t) = F_1(t) - F_2(t)$, indica la \textbf{differenza tra le forze di propulsione} $F_1(t), F_2(t)$ applicate sul drone.

\resource{0.5}{drone.png}{Rappresentazione nel piano del drone considerato.}

\section{Espressione del sistema in forma di stato e calcolo del sistema linearizzato intorno ad una coppia di equilibrio}

Innanzitutto, esprimiamo il sistema \eqref{eq:system} nella seguente forma di stato
\begin{subequations}
\begin{align}\label{eq:state_form}
	\dot{x} &= f(x,u) \\
	y &= h(x,u)
\end{align}
\end{subequations}
andando ad individuare stato $x$, ingresso $u$ e uscita $y$:
%
\begin{align*}
	x := \dots, \quad u := \dots, \quad y := \dots.
\end{align*}
%
Coerentemente con questa scelta, ricaviamo dal sistema~\eqref{eq:system} la seguente espressione per le funzioni $f$ ed $h$
%
\begin{align*}
	f(x,u) &:= \dots
	\\
	h(x,u) &:= \dots.
\end{align*}
%
Una volta calcolate $f$ ed $h$ esprimiamo~\eqref{eq:system} nella seguente forma di stato
%
\begin{subequations}\label{eq:our_system_state_form}
\begin{align}
	\begin{bmatrix}
		\dot{x}_1
		\\
		\dots
	\end{bmatrix} &= \dots \label{eq:state_form_1}
	\\
	y &= \dots.
\end{align}
\end{subequations}
%
Per trovare la coppia di equilibrio $(x_e, u_e)$ di~\eqref{eq:our_system_state_form}, andiamo a risolvere il seguente sistema di equazioni
%
\begin{align}
	\dots,
\end{align}
%
dal quale otteniamo
%
\begin{align}
	x_e := \dots,  \quad u_e = \dots.\label{eq:equilibirum_pair}
\end{align}
%
Definiamo le variabili alle variazioni $\delta x$, $\delta u$ e $\delta y$ come 
%
\begin{align*}
	\delta x &= \dots, 
	\quad
	\delta u = \dots, 
	\quad
	\delta y = \dots.
\end{align*}
%
L'evoluzione del sistema espressa nelle variabili alle variazioni pu\`o essere approssimativamente descritta mediante il seguente sistema lineare
%
\begin{subequations}\label{eq:linearized_system}
\begin{align}
	\delta \dot{x} &= A\delta x + B\delta u
	\\
	\delta y &= C\delta x + D\delta u,
\end{align}
\end{subequations}
%
dove le matrici $A$, $B$, $C$ e $D$ vengono calcolate come
%
\begin{subequations}\label{eq:matrices}
\begin{align}
	A &= \dots
	\\
	B &= \dots
	\\
	C &= \dots
	\\
	D &= \dots.
\end{align}
\end{subequations}
%
\section{Calcolo Funzione di Trasferimento}

In questa sezione, andiamo a calcolare la funzione di trasferimento $G(s)$ dall'ingresso $\delta u$ all'uscita $\delta y$ mediante la seguente formula 
%
%
\begin{align}\label{eq:transfer_function}
G(s) = \dots = \dots.
\end{align}
%
Dunque il sistema linearizzato~\eqref{eq:linearized_system} è caratterizzato dalla funzione di trasferimento~\eqref{eq:transfer_function} con $\dots$ poli $p_1 = \cdots, \cdots$ e $\dots$ zeri $z_i =\cdots$. In Figura \dots mostriamo il corrispondente diagramma di Bode. 

\dots

\section{Mappatura specifiche del regolatore}
\label{sec:specifications}

Le specifiche da soddisfare sono
\begin{itemize}
	\item[1)] \dots\\
	\item[2)] \dots\\
	....\\
	\item[6)] \dots.
\end{itemize}
%
Andiamo ad effettuare la mappatura punto per punto le specifiche richieste. \dots  

Pertanto, in Figura \dots, mostriamo il diagramma di Bode della funzione di trasferimento $G(s)$ con le zone proibite emerse dalla mappatura delle specifiche.

\dots

\section{Sintesi del regolatore statico}
\label{sec:static_regulator}

In questa sezione progettiamo il regolatore statico $R_s(s)$ partendo dalle analisi fatte in sezione~\ref{sec:specifications}.

\dots

Dunque, definiamo la funzione estesa $G_e(s) = R_s(s)G(s)$ e, in Figura \dots, mostriamo il suo diagramma di Bode per verificare se e quali zone proibite vengono attraversate.

\dots

Da Figura \dots, emerge \dots


\section{Sintesi del regolatore dinamico}

In questa sezione, progettiamo il regolatore dinamico $R_d(s)$. 
%
Dalle analisi fatte in Sezione~\ref{sec:static_regulator}, notiamo di essere nello Scenario di tipo \dots. Dunque, progettiamo $R_d(s)$ riccorrendo a \dots


In Figura \dots, mostriamo il diagramma di Bode della funzione d'anello $L(s) = R_d(s) G_e(s)$

\dots

\section{Test sul sistema linearizzato}

In questa sezione, testiamo l'efficacia del controllore progettato sul sistema linearizzato con \dots

\section{Test sul sistema non lineare}

In questa sezione, testiamo l'efficacia del controllore progettato sul modello non lineare con \dots


\section{Punti opzionali}

\subsection{Primo punto}

\dots 

\subsection{Secondo punto}

\dots

\subsection{Terzo punto}

\dots

\section{Conclusioni}

\dots

\end{document}
