\usepackage[margin=3cm]{geometry}
\geometry{a4paper, top=2cm, bottom=3cm, left=1.5cm, right=1.5cm, heightrounded, bindingoffset=5mm}

\usepackage[T1]{fontenc}
\usepackage[utf8]{inputenc}
\usepackage[italian]{babel}

\usepackage{psfrag,amsmath,amssymb,amsthm,amsfonts,verbatim}
\usepackage{mathtools, physics, textcomp}
\usepackage[figurename=Fig.,font=small,labelfont=bf]{caption}
\usepackage{capt-of, subcaption, wrapfig, float, hyperref, cleveref}

\usepackage{latexsym, gensymb, graphicx, graphics}
\usepackage{xcolor,color,soul}
\usepackage{fancyhdr}
\usepackage{indentfirst}
\usepackage{newlfont}
\usepackage{footnote}

\setlength{\parindent}{0pt}

%\usepackage{tikz}
%\usetikzlibrary{snakes}
%\usetikzlibrary{positioning}
%\usetikzlibrary{shapes,arrows}

%\tikzstyle{block} = [draw, fill=white, rectangle, 
%	minimum height=3em, minimum width=6em]
%\tikzstyle{sum} = [draw, fill=white, circle, node distance=1cm]
%\tikzstyle{input} = [coordinate]
%\tikzstyle{output} = [coordinate]
%\tikzstyle{pinstyle} = [pin edge={to-,thin,black}]

\graphicspath{ {./assets} }

\newcommand{\bb}{
	\bigbreak
}

\newcommand{\R}{
	\ensuremath
	\mathbb{R}
}

\newcommand{\C}{
	\ensuremath
	\mathbb{C}
}

\newcommand{\Lap}{
	\ensuremath
	\mathcal{L}
}

\newcommand{\M}[3]{
	\ensuremath
	\mathcal{M}\paren{#1,#2,#3}
}

\newcommand{\llimit}[3]{
	\ensuremath
	\lim_{#1 \rightarrow #2} #3
}

\newcommand{\pinf}{
	\ensuremath
	+\infty
}

\newcommand{\minf}{
	\ensuremath
	-\infty
}

\newcommand{\intreal}[2]{
	\ensuremath
	\int_{\R} #1 \dd{#2}
}

\newcommand{\intnoreal}[4]{
	\ensuremath
	\int_{#1}^{#2} #3 \dd{#4}
}

\newcommand{\ifincases}[1]{
	\ensuremath
	\; \textrm{ se } \; #1
}

\newcommand{\sgn}[1]{
	\operatorname{sgn} \left( #1 \right)
}

\newcommand{\paren}[1]{
	\ensuremath
	\left( #1 \right)
}

\newcommand{\sparen}[1]{
	\ensuremath
	\left[ #1 \right]
}

\newcommand{\resource}[3]{
	\begin{center}
	\includegraphics[scale = #1]{#2}
	\end{center}
	\captionof{figure}{#3}
}

\newcommand{\rarr}{
	$\longrightarrow$
}

\newcommand{\lt}[1]{
	\ensuremath
	\mathcal{L}\sparen{#1}
}

\newcommand{\lat}[1]{
	\ensuremath
	\mathcal{L}^{-1}\sparen{#1}
}

\newcommand{\ltint}[3]{
	\ensuremath
	\int_{0^-}^{\pinf} #1 e^{#2} \dd{#3}
}

\newcommand{\latint}[3]{
	\ensuremath
	\frac{1}{2\pi j}\int_{\sigma-j\infty}^{\sigma + j\infty} #1 e^{#2} \dd{#3}
}

\newcommand{\cof}[1]{
	\ensuremath
	\textrm{Cof}\paren{#1}
}